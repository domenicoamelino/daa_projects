\documentclass[tesi.tex]{subfiles}
\graphicspath{{images/}{images/noChallenge/}{images/withChallenge/}}

%\usepackage[english]{babel}
\selectlanguage{english}%

\makeatletter% Set distance from top of page to first float
\setlength{\@fptop}{5pt}
\makeatother

\begin{document}
\chapter*{\addcontentsline{toc}{chapter}{A: Direct Anonymous Attestation }Appendix A: Direct Anonymous Attestation}

A few solutions are focused on integrating DAA in existing trusted platforms, like ARM TrustZone,
supporting security-critical services in microprocessor platforms.
TrustZone, in particular, provides a secure area in the main processor, the Trusted Execution Enviroment (TEE),
guaranteeing that the code and data loaded inside that area are protected:
when a user invokes a secure operation, the system switches from the normal environment, called Rich Execution Environment (REE), to TEE.
%
While TrustZone does not support the DAA in itself, a few DAA proposals for mobile devices are built on top ot its architecture.
Wachsmann et al~\cite{wachsmann2010lightweight} propose a lightweight anonymous authentication scheme
without linkability for embedded device exploiting TrustZone primitives.
% However, it does not allow user to choice the level of required security.
In~\cite{zhang2014mdaak} Zhang et al. proposed \emph{Mdaak}, a DAA framework for mobile platform
and, similarly, Yang et al.~\cite{yang2014lightweight} propose a detailed infrastructure for four ECC-based DAA implementations.
Nevertheless, neither work provides any protection mechanism for DAA data and, further,
the overhead introduced by TrustZone TEE/REE context switching is ignored.

In~\cite{yang2015daa} a DAA scheme based on TrustZone functionalities is realized.
Additionally, this solution relies on SRAM PUF as a root of trust,
considers the pre-computation of heavier setup values in order to realize a quite expensive execution,
and further reduces the context switch overhead between TEE and REE.
% However, the work does not describe any practical application context. 
%
In addition to the previous proposals, the technical literature offers several comparisons among different DAA schemes
useful to identify the best DAA scheme choice for a specific infrastructure and domain~\cite{xi2014direct,chen2010design}.


\end{document}